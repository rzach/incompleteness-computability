%\olimport*[first-order-logic]{first-order-logic}
% Part: first-order-logic
% Chapter: syntax-and-semantics

\chapter{First-order Logic}

\olimport*[first-order-logic/syntax-and-semantics]{first-order-languages}

\olimport*[first-order-logic/syntax-and-semantics]{terms-formulas}

\olimport*[first-order-logic/syntax-and-semantics]{free-vars-sentences}

\olimport*[first-order-logic/syntax-and-semantics]{substitution}

\olimport*[first-order-logic/syntax-and-semantics]{structures}

\olimport*[first-order-logic/syntax-and-semantics]{satisfaction}

\olimport*[first-order-logic/syntax-and-semantics]{semantic-notions}

\section{Theories}

\begin{defn}
A set of !!{sentence}s~$\Gamma$ is \emph{closed} iff, whenever
$\Gamma \Entails !A$ then $!A \in \Gamma$.  The \emph{closure} of a set
of !!{sentence}s~$\Gamma$ is $\Setabs{!A}{\Gamma \Entails !A}$.

We say that~$\Gamma$ is \emph{axiomatized by} a set of
sentences~$\Delta$ if $\Gamma$ is the closure of~$\Delta$
\end{defn}

\begin{ex}
The theory of strict linear orders in the language~$\Lang L_<$ is
axiomatized by the set
\begin{align*}
& \lforall[x][\lnot x < x], \\
& \lforall[x][\lforall[y][((x < y \lor y <
    x) \lor x = y)]], \\
& \lforall[x][\lforall[y][\lforall[z][((x < y
      \land y < z) \lif x < z)]]]
\end{align*}
It completely captures the intended !!{structure}s: every strict
linear order is a model of this axiom system, and vice versa, if $R$
is a linear order on a set $X$, then the structure $\Struct M$ with
$\Domain M = X$ and $\Assign{<}{M} = R$ is a model of this theory.
\end{ex}

\begin{ex}
The theory of groups in the language $\Obj 1$ (!!{constant}), $\cdot$
(two-place !!{function}) is axiomatized by
\begin{align*}
& \lforall[x][(x \cdot \Obj 1) = x]\\
& \lforall[x][\lforall[y][\lforall[z][\eq[(x \cdot (y \cdot z))][((x
          \cdot y) \cdot z)]]]]\\
& \lforall[x][\lexists[y][(x \cdot y) = \Obj 1]]
\end{align*}
\end{ex}

\begin{ex}
The theory of Peano arithmetic is axiomatized by the following
sentences in the language of arithmetic~$\Lang L_A$.
\begin{align*}
& \lnot\lexists[x][x' = \Obj 0]\\
& \lforall[x][\lforall[y][(x' = y' \lif x = y)]]\\
& \lforall[x][\lforall[y][(x < y \liff \lexists[z][(x + z' = y)])]]\\
& \lforall[x][\eq[(x + \Obj 0)][x]]\\
& \lforall[x][\lforall[y][\eq[(x + y')][(x + y)']]]\\
& \lforall[x][\eq[(x \times \Obj 0)][\Obj 0]]\\
& \lforall[x][\lforall[y][\eq[(x \times y')][((x \times y) + x)]]]\\
\intertext{plus all sentences of the form}
& (!A(\Obj 0) \land \lforall[x][(!A(x) \lif !A(x'))]) \lif \lforall[x][!A(x)]
\end{align*}
Since there are infinitely many sentences of the latter form, this
axiom system is infinite.  The latter form is called the
\emph{induction schema}. (Actually, the induction schema is a bit more
complicated than we let on here.)

The third axiom is an \emph{explicit definition} of~$<$.
\end{ex}


\OLEndChapterHook

\chapter{Natural Deduction}

\olimport*[first-order-logic/natural-deduction]{introduction}

\olimport*[first-order-logic/natural-deduction]{rules-and-proofs}

\olimport*[first-order-logic/natural-deduction]{proving-things}

\olimport*[first-order-logic/natural-deduction]{identity}

\olimport*[first-order-logic/natural-deduction]{proof-theoretic-notions}

\OLEndChapterHook


