% !TeX root = ../ic-screen.tex

\chapter{About this Book}

This is a textbook on G\"odel's incompleteness theorems and recursive
function theory. I use it as the main text when I teach Philosophy 479
(Logic III) at the University of Calgary. It is based on material from the
\href{https://openlogicproject.org}{Open Logic Project}.

As its name suggests, the course is the third in a sequence, so
students (and hence readers of this book) are expected to be familiar
with first-order logic already.  (Logic~I uses the text
\href{https://forallx.openlogicproject.org}{\emph{forall x: Calgary}},
and Logic~II another textbook based on the OLP,
\href{https://slc.openlogicproject.org}{\emph{Sets, Logic,
Computation}}.) The material assumed from Logic~II, however, is
included as \cref{fol:chap,nd:chap}.

Logic III is a thirteen-week course, meeting three hours per week.
This is typically enough to cover the material in
\cref{inc:int::chap,cmp:rec::chap,inc:art::chap,inc:req::chap,inc:inp::chap}
and either \cref{mod:chap} or \cref{lambda:chap}, depending on student
interest. You may want to spend more time on the basics of
first-order logic and especially on natural deduction, if students are
not already familiar with it. Note that when provability in
arithmetical theories (such as $\Th{Q}$ and $\Th{PA}$) is discussed in
the main text, the proofs of provability claims are not given using a
specific !!{derivation} system. Rather, that certain claims follow from the
axioms by first-order logic is justified intuitively. However,
\cref{deriv:chap} contains a number of examples of actual natural
deduction !!{derivation}s from the axioms of~$\Th{Q}$.

\section*{Acknowledgments}

The material in the OLP used in
\cref{inc:int::chap,cmp:rec::chap,inc:art::chap,inc:req::chap,inc:inp::chap,lambda:chap}
was based originally on Jeremy Avigad's lecture notes on
``Computability and Incompleteness,'' which he contributed to the OLP.
I have heavily revised and expanded this material. The lecture notes,
e.g., based theories of arithmetic on an axiomatic !!{derivation} system. Here,
we use Gentzen's standard natural deduction system (described in
\cref{nd:chap}), which requires dealing with trees primitive
recursively (in \cref{cmp:rec:tre:sec}) and a more complicated
approach to the arithmetization of !!{derivation}s (in
\cref{inc:art:pnd:sec}). The material in \cref{lambda:chap} was also
expanded by Zesen Qian during his stay in Calgary as a Mitacs summer
intern.

The material in the OLP on model theory and models of arithmetic in
\cref{mod:chap} was originally taken from Aldo Antonelli's lecture
notes on ``The Completeness of Classical Propositional and Predicate
Logic,'' which he contributed to the OLP before his untimely death in
2015.

The biographies of logicians in \cref{bios:chap} and much of the
material in \cref{nd:chap} are originally due to Samara Burns.
Dana H\"agg originally worked on the material in \cref{fol:chap}.
