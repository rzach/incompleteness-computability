\textbf{Second-order logic} is an extension of first-order logic by
variables for relations and functions, which can be
quantified. Structures for second-order logic are just like
first-order structures and give the interpretations of all non-logical
symbols of the language.  Variable assignments, however, also assign
relations and functions on the domain to the second-order
variables. The satisfaction relation is defined for second-order
!!{formula}s just like in the first-order case, but extended to deal
with second-order variables and quantifiers.

Second-order quantifiers make second-order logic \textbf{more
  expressive} than first-order logic.  For instance, the identity
relation on the domain of a structure can be defined without~$\eq$, by
$\lforall[X][(X(x) \liff X(y))]$. Second-order logic can express the
\textbf{transitive closure} of a relation, which is not expressible in
first-order logic.  Second-order quantifiers can also express
properties of the domain, that it is finite or infinite,
!!{enumerable} or !!{nonenumerable}. This means that, e.g., there is a
second-order !!{sentence}~$\fn{Inf}$ such that $\Sat{M}{\fn{Inf}}$ iff
$\Domain{M}$ is infinite. Importantly, these are \textbf{pure}
second-order sentences, i.e., they contain no non-logical
symbols. Because of the compactness and L\"owenheim-Skolem theorems,
there are no first-order sentences that have these properties.  It
also shows that the \textbf{compactness and L\"owenheim-Skolem
  theorems fail for second-order logic}.

Second-order quantification also makes it possible to replace
first-order schemas by single sentences. For instance,
\textbf{second-order arithmetic}~$\Th{PA^2}$ is comprised of the
axioms of~$\Th{Q}$ plus the single \textbf{induction axiom}
\[
\lforall[X][((X(\Obj 0) \land \lforall[x][(X(x) \lif X(x'))]) \lif
  \lforall[x][X(x)])].
\]
In contrast to first-order~$\Th{PA}$, all second-order models
of~$\Th{PA^2}$ are isomorphic to the standard model. In other words,
$\Th{PA^2}$ has \textbf{no non-standard models}. 

Since second-order logic includes first-order logic, it is
undecidable. First-order logic is at least axiomatizable, i.e., it has
a sound and complete !!{derivation} system. Second-order logic does not, it is
\textbf{not axiomatizable}. Thus, the set of validities of
second-order logic is highly non-computable. In fact, pure second-order
logic can express set-theoretic claims like the \textbf{continuum
  hypothesis}, which are independent of set theory.
