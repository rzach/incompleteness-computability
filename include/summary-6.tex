The \textbf{partial recursive functions} are those that can be defined
from the basic functions $\Zero$, $\Succ$, $\Proj{i}{j}$ using
composition, primitive recursion, and unbounded search. Such
definitions can themselves be encoded as numbers. When a suitable
coding is fixed and a partial recursive function~$f$ has code~$e$ we
call $e$ an \textbf{index} of~$f$ and denote $f$ also by~$\cfind{e}$.
Not just definitions of functions, but also computations of functions
applied to arguments can be coded as numbers. The property of deciding
whether $s$ codes the computation of $\cfind{e}$ applied to argument
$x$, $T(e, x, s)$, and the function that given a code~$s$ of a
computation as input returns the output of that computation, $U(s)$,
are both primitive recursive. Thus, we have $\cfind{e}(x) \simeq
U(\umin{s}{T(e,x,s)})$. This is the \textbf{Kleene normal form
theorem}.

The Kleene noremal form theorem actually applies not just to partial
recursive functions, but to any known model of computability. E.g., it
is also true for Turing machines that their definitions and
computations can be coded as numbers in such a way that the Kleene
normal form theorem holds. The indexing of partial computable
functions via the Kleene normal form theorem thus enables a very
general investigation of computable functions. This area of research is
called \textbf{computability theory}. Among its results are that there
is a universal two-place partial recursive function~$\fn{Un}$ such
that $\fn{Un}(e,x) \simeq \cfind{e}(x)$, that there is no total
universal computable function, and that the question whether
$\cfind{e}(x)$ is defined is undecidable (the \textbf{Halting
problem}). The \textbf{self-halting set}~$K = \Setabs{e}{\cfind{e}(e)
\text{ is defined}}$ is another example of a set of natural numbers
that is undecidable. It is \textbf{computably enumerable}, however,
i.e., it is the range of a computable function.

Via G\"odel numbering, computability theory can be used to give very
general formulations of G\"odel's theorems about undecidability and
incompleteness of theories. We can prove, using results from
computability theory, that every $\omega$-consistent theory that
represents all primitive recursive relations is undecidable, and that
when such a theory is computably enumerable, it must be incomplete.