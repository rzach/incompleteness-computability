% Part: incompleteness
% Chapter: computability-incompleteness
% Section: introduction

\documentclass[../../../include/open-logic-section]{subfiles}

\begin{document}

\olfileid{inc}{cpi}{int}
\olsection{Computable Enumerability and Axiomatizable Theories}

There is a close connection between the incompleteness theorems and
computability. This should be apparent from the fact that in order to
obtain the proof-theoretic versions of the first and second
incompleteness theorems, we had to first develop a fair bit of
computable function theory. We had to define primitive recursive
functions, show that we can arithmetize provability using primitive
recursive functions, and show that $\Th{Q}$ represents all primitive
recursive functions and relations, in particular the proof
relation~$\Prf[\Th{T}]$ of any axiomatizable theory~$\Th{T}$. Two of the
assumptions of the incompleteness theorem are essentially assumptions
about computability. The assumption that $\Th{T}$ is axiomatizable
means that $\Th{T}$ is axiomatized by a \emph{decidable} set of
axioms. The assumption that $\Th{T}$ extends $\Th{Q}$ was needed to
guarantee that $\Th{T}$ represents all primitive recursive functions
and relations.

In fact, we can generalize the incompleteness theorems by making use
of ideas from computability theory. This requires only that our model
of computation can handle !!{formula}s and proofs. The arithmetization of
provability shows that models of computation that directly only apply
to numbers can do this---via G\"odel numbering. Other models of
computability may not require as much work, and can avoid all the
trouble of coding sequences, !!{formula}s, and derivations we went through
to show that $\Prf[\Th{T}]$ is primitive recursive. For instance, a
suitably general notion of Turing machine would allow us to represent
!!{formula}s and proofs directly as sequences of symbols by including
symbols for logical connectives and quantifiers, !!{variable}s,
!!{predicate}s, etc., as symbols that are allowed on the tape.
\emph{Some} coding will still be necessary, as Turing machines only
allow finitely many symbols in their alphabet, but there are
infinitely many !!{variable}s, !!{constant}s, etc. In fact, to verify
all the details required for showing that there is a Turing machine
that decides the proof relation would be much more complicated than
showing that it (understood as a relation of G\"odel numbers) is
primitive recursive.

The assumption that $\Th{T}$ is axiomatizable can be generalized to
the assumption that the set of theorems of~$\Th{T}$ is !!{computably
enumerable}. If our model of computability requires G\"odel numbering
(e.g., the partial recursive functions), then this amounts to saying
that 
\[
  \Gn{\Th{T}} = \Setabs{\Gn{!A}}{\Th{T} \Proves !A}
\]
is !!{c.e.} This is implied by the assumption that $\Th{T}$ is
axiomatizable.

\begin{prop}\ollabel{prop:ax-theory-ce} Suppose $\Th{T}$ is
axiomatized by a decidable set of axioms~$\Gamma$. Then $\Th{T}$ is
!!{c.e.}
\end{prop}

\begin{proof}
  Informally, we can computably enumerate all theorems of~$\Th{T}$ by
  searching through all possible proofs in the language
  of~$\Th{T}$ in whatever proof system we prefer. Acceptable proof
  systems have the property that it can be decided if an arrangements
  of symbols and !!{formula}s is a correct proof, e.g., testing whether
  a given putative application of an inference rule in natural deduction is
  correct is decidable. The assumption that $\Gamma$ is decidable
  allows us to check if a correct derivation is a derivation
  \emph{from}~$\Gamma$. E.g., we can check, for each open assumption
  of a natural deduction derivation, that it is an axiom in~$\Gamma$.
  If a derivation passes these checks, we output the end-!!{formula}; it
  is a theorem of~$\Th{T}$.

  More formally, we can proceed as follows. If $\Gn{\Gamma}$ is
  primitive recursive, \tagrefs{prfAX/{inc:art:pax:prop:prf-prim-rec},
  prfSC/{inc:art:plk:prop:prf-prim-rec},
  prfND/{inc:art:pnd:prop:prf-prim-rec}} shows
  that the proof predicate $\Prf[\Gamma]$ is primitive recursive. It
  takes little effort to verify that the same proof shows that if
  $\Gn{\Gamma}$ is computable then $\Prf[\Gamma]$ is computable. Let
  $f(y) = \umin{x}{\Prf[\Gamma](x,y)}$. Obviously, $f(y)\fdefined$ iff
  $y$ is the G\"odel number of a theorem of~$\Th{T}$. Thus,
  $\Gn{\Th{T}}$ is the domain of a partial computable function and so
  !!{c.e.} by \olref[cmp][thy][eqc]{thm:ce-equiv}.
\end{proof}

It is perhaps surprising that the converse of
\olref{prop:ax-theory-ce} is also true: every !!{c.e.} theory is
axiomatizable. This is proved using what's known as ``Craig's trick.''

\begin{prop}\ollabel{prop:ce-ax}
If\/ $\Th{T}$ is !!a{c.e.} theory, then $\Th{T}$ is axiomatizable by a
decidable (in fact, primitive recursive) set of axioms.
\end{prop}

\begin{proof}
  Since $\Th{T}$ is !!{c.e.}, there is a computable enumeration~$f$ of
  it. We may assume that the enumeration is primitive recursive by
  \olref[cmp][thy][eqc]{thm:ce-equiv},
  case~\olref[cmp][thy][eqc]{case:ran-prim}. Let $!A_n$ be the $n$-th
  element of this enumeration. Consider the
  set
  \[
  \Gamma = \{!A_0, !A_1 \land !A_1, !A_2 \land (!A_2 \land !A_2), \dots\}
  \]
  $\Gamma$ clearly axiomatizes~$\Th{T}$, since every theorem
  of~$\Th{T}$ appears as some $!A_n$ in the enumeration, and follows
  from $!A_n \land \dots \land !A_n$ ($n+1$ copies of $!A_n$), which
  is the $(n+1)$-st element of $\Gamma$. $\Gamma$ is also decidable: to
  test if $!A \in \Gamma$, determine if it is of the form $!B \land
  (!B \land \dots \land !B)\dots)$, and if so, how many conjuncts
  of~$!B$ it contains. Let $n$ be that number. If $!A$ is not a
  conjunction of identical conjuncts, let $n=0$ and $!B \ident !A$.
  Then $!A \in \Gamma$ iff $!B \ident !A_{n}$

  More formally, we can proceed as follows. Let  $f(n) = \Gn{!A_n}$ be
  a primitive recursive enumeration of~$\Th{T}$. Verify that the
  functions~$i$ and $b$ such that $i(\Gn{A}) = n$ and $b(\Gn{A}) =
  \Gn{B}$ are primitive recursive.  $!A \in \Gamma$ iff $b(\Gn{A}) =
  f(i(\Gn{A}))$. The latter expression defines a primitive recursive
  predicate.
\end{proof}

\begin{prob}
  Assume $f$ is a primitive recursive enumeration of some set of
  G\"odel numbers of !!{formula}s~$\Gamma$. Verify that the functions $b$
  and $i$ in the proof of \olref{prop:ce-ax} are primitive recursive.
\end{prob}

\end{document}
