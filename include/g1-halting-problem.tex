% Part: incompleteness
% Chapter: computability-incompleteness
% Section: g1-halting-problem

\documentclass[../../../include/open-logic-section]{subfiles}

\begin{document}

\olfileid{inc}{cpi}{hal}
\olsection{Incompleteness via the Halting Problem}

\begin{thm}\ollabel{thm:undecidable}
If $\Th{T}$ is an $\omega$-consistent theory that represents all
primitive recursive relations, then $\Th{T}$~is undecidable.
\end{thm}

\begin{proof}
  Since $\Th{T}$ represents all primitive recursive functions, it
  represents in particular Kleene's $T$ predicate. That is, there is a
  !!{formula}~$!T(e, x, s)$ such that:
  \begin{enumerate}
    \item\ollabel{case:def} if $\cfind{e}(n) \fdefined$ there is a~$k \in \Nat$ such
    that $\Th{T} \Proves !T(\num{e}, \num{n}, \num{k})$, and
    \item\ollabel{case:undef} if $\cfind{e}(n) \fundefined$, then for all $k \in \Nat$,
    $\Th{T} \Proves \lnot !T(\num{e}, \num{n}, \num{k})$.
  \end{enumerate}
  We will show that if $\Th{T}$ is $\omega$-consistent and represents
  all primitive recursive relations, we would be able to decide the
  set $K = \Setabs{e}{\cfind{e}(e) \fdefined}$, the ``self-halting
  problem.'' 

  Since $\Th{T}$ is a theory, it is closed under consequence. In
  particular if $\Th{T} \Proves !T(\num{e}, \num{e}, \num{k})$, then
  also $\Th{T} \Proves \lexists[z][!T(\num{e}, \num{e}, z)]$. Together
  with~\olref{case:def}, this means that if $\cfind{e}(e) \fdefined$,
  then $\Th{T} \Proves \lexists[z][!T(\num{e}, \num{e}, z)]$. On the
  other hand, as $\Th{T}$ is $\omega$-consistent, we cannot have both
  that $\Th{T} \Proves \lnot !T(\num{e}, \num{e}, \num{k})$ for all $k
  \in \Nat$ and also $\Th{T} \Proves \lexists[z][!T(\num{e}, \num{e},
  z)]$. So, by \olref{case:undef}, if $\cfind{e}(e) \fundefined$, then $\Th{T} \Proves/
  \lexists[z][!T(\num{e}, \num{n}, z)]$. Together, we have that
  $\cfind{e}(e) \fdefined$ iff $\Th{T} \Proves \lexists[z][!T(\num{e},
  \num{n}, z)]$.

  If $\Th{T}$ were decidable, then this would allow us to answer
  whether $\cfind{e}(e)$ is defined, i.e., if $e \in K$, by deciding
  whether $\Th{T} \Proves \lexists[z][!T(\num{e}, \num{e}, z)]$.
  However, $K$ is undecidable (\olref[cmp][thy][ncp]{thm:K}).
\end{proof}

\begin{thm}
  \ollabel{thm:complete-decidable}
  If\/ $\Th{T}$ is !!a{complete} consistent !!{c.e.} theory, then
  $\Th{T}$ is decidable.
\end{thm}

\begin{proof}
  Informally, we can computably enumerate all theorems of !!a{c.e.}
  theory. To decide if $\Th{T} \Proves !A$ for a given $!A$, just
  search through this enumeration of all theorems until we find either
  $!A$ or~$\lnot !A$. Since $\Th{T}$ is !!{complete}, this must
  eventually happen. When $!A$ shows up in the enumeration we
  know that $\Th{T} \Proves !A$. Consistency ensures that only one of $!A$ or
  $\lnot !A$ is a theorem. This, when $\lnot !A$ shows up in the enumeration we
  know that $\Th{T} \Proves/ !A$.

  More formally, since $\Th{T}$ is complete, consistent, and
  enumerated by a computable function~$f$, the
  complement of the set of G\"odel numbers of theorems of $\Th{T}$ is
  \[\Complement{\Gn{\Th{T}}} = \Setabs{n}{\lexists[k][(\lnot
  \fn{Sent}(n) \lor (\Gn{\lnot} \concat f(k)) = n)]},\] ($\Gn{!A} \in
  \Nat \setminus \Gn{\Th{T}}$ iff $!A$ is not a sentence at all or its
  negation is a theorem of~$\Th{T}$. By
  \olref[cmp][thy][eqc]{thm:exists-char}, $\Complement{\Gn{\Th{T}}}$
  is !!{c.e.}, and so by \olref[cmp][thy][cmp]{thm:ce-comp}, $\Th{T}$
  is decidable.
\end{proof}

\begin{cor}
  If\/ $\Th{T}$ is $\omega$-consistent, !!{c.e.} and represents all
  primitive recursive relations, it is incomplete.
\end{cor}

\begin{proof}
  Suppose $\Th{T}$ is complete. Since $\omega$-consistent theories
  are consistent, $\Th{T}$ is consistent and decidable by
  \olref{thm:complete-decidable}. This contradicts
  \olref{thm:undecidable}.
\end{proof}

This proof of G\"odel's first incompleteness theorem does not provide
an example of an undecidable sentence the way a proof via the diagonal
lemma does (see \olref[inc][inp][1in]{sec}).

\begin{thm}\ollabel{thm:g1} Suppose $\Th{T}$ is an $\omega$-consistent
  !!{c.e.} theory that represents all primitive recursive relations.
  Then there is !!a{sentence}~$G$ such that $\Th{T} \Proves/ G$ and
  $\Th{T} \Proves/ \lnot G$. $G$~can be specified explicitly from the
  computable enumeration of\/~$\Th{T}$.
\end{thm}

\begin{proof}
  Let $f$ be a computable enumeration of~$\Th{T}$. Consider the
  function~$g(e)$ which returns $0$ if a search for a proof of
  $\lnot\lexists[z][!T(\num{e}, \num{e}, z)]$ using~$f$ is successful,
  and is undefined if it is not. Let $k$ be the index of~$g$, i.e., $g
  = \cfind{k}$. Then $g(e) \fdefined$ iff $\Th{T} \Proves
  \lnot\lexists[z][!T(\num{e}, \num{e}, z)]$ by $g$'s definition.
  Consequently there is an $m$ such that $T(k, e, m)$ iff $\Th{T}
  \Proves \lnot\lexists[z][!T(\num{e}, \num{e}, z)]$. Taking $e = k$,
  we have that there is an $m$ such that $T(k, k, m)$ iff $\Th{T}
  \Proves \lnot\lexists[z][!T(\num{k}, \num{k}, z)]$.

  Suppose that $g(k) \fdefined$, i.e., $\cfind{k}(k) \fdefined$, i.e.,
  for some $m$, $T(k, k, m)$ holds. By the preceding equivalence, we
  have that $\Th{T} \Proves \lnot\lexists[z][!T(\num{k}, \num{k},
  z)]$. On the other hand, since $\Th{T}$ represents~$T$, we have that
  $\Th{T} \Proves !T(\num{k}, \num{k}, \num{m})$, and so also $\Th{T}
  \Proves \lexists[z][!T(\num{k}, \num{k}, z)]$. This contradicts the
  consistency of~$\Th{T}$. We can conclude that $g(k) \fundefined$,
  i.e., $\Th{T} \Proves/ \lnot\lexists[z][!T(\num{k}, \num{k}, z)]$.

  Since $g(k) \fundefined$, $T(k, k, m)$ holds for no~$m$. Since
  $\Th{T}$ represents~$T$, we have that $\Th{T} \Proves \lnot
  !T(\num{k}, \num{k}, \num{m})$ for all~$m$. Since $\Th{T}$ is
  $\omega$-consistent, $\Th{T} \Proves/ \lexists[z][!T(\num{k},
  \num{k}, z)]$.

  We've shown that we can the sentence $G$ to be
  $\lnot\lexists[z][!T(\num{k}, \num{k}, z)]$. If we have an explicit
  description of~$g$, we can determine the index~$k$ of~$g$. Together
  with an explicit description of~$T$ and an effective proof of the
  representability of~$T$ in~$\Th{T}$ as !!a{formula}~$!T$, we have
  that $G$ can be found effectively from the computable enumeration
  of~$\Th{T}$.
\end{proof}

Remember that $\lexists[z][!T(\num{e}, \num{k}, z)]$ ``says that''
$\cfind{e}(k)\fdefined$. Consequently, $\lexists[z][!T(\num{k},
\num{k}, z)]$ ``says that'' $g(k) = \cfind{k}(k)\fdefined$, or that
$g$ self-halts. In other words then, $g$ searches for a proof
in~$\Th{T}$ of a formula that says ``$g$ does not self-halt'' and
halts if such a proof exists, but doesn't halt if it doesn't. 

\end{document}
