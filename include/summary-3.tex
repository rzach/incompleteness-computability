The proof of the incompleteness theorems requires that we have a way
to talk about provability in a theory (such as $\Th{PA}$) in the
language of the theory itself, i.e., in the language of
arithmetic. But the language of arithmetic only deals with numbers,
not with formulas or !!{derivation}s.  The solution to this problem is
to define a systematic mapping from !!{formula}s and derivations to
numbers. The number associated with !!a{formula} or a !!{derivation}
is called its \textbf{G\"odel number}.  If $!A$ is !!a{formula},
$\Gn{!A}$ is its G\"odel number. We showed that important operations on
!!{formula}s turn into primitive
recursive functions on the respective G\"odel
numbers. For instance, $\Subst{!A}{t}{x}$, the operation of
substituting a term~$t$ for every free occurrence of~$x$ in~$!A$,
corresponds to an arithmetical function $\fn{subst}(n,m,k)$ which, if
applied to the G\"odel numbers of $!A$, $t$, and $x$, yields the
G\"odel number of~$\Subst{!A}{t}{x}$. In other words,
$\fn{subst}(\Gn{!A}, \Gn{t}, \Gn{x}) = \Gn{\Subst{!A}{t}{x}}$.
Likewise, properties of !!{derivation}s
turn into primitive recursive relations on the respective G\"odel
numbers.  In particular, the property $\fn{Deriv}(n)$ that holds
of~$n$ if it is the G\"odel number of a correct !!{derivation} in
natural deduction, is primitive recursive.  Showing that these are
primitive recursive required a fair amount of work, and at times some
ingenuity, and depended essentially on the fact that operating with
sequences is primitive recursive.  If a theory~$\Th{T}$ is
!!{decidable}, then we can use $\fn{Deriv}$ to define !!a{decidable}
relation $\Prf[\Th{T}](n, m)$ which holds if $n$ is the G\"odel number
of !!a{derivation} of the !!{sentence} with G\"odel number~$m$
from~$\Th{T}$.  This relation is primitive recursive if the set of
axioms of~$\Th{T}$ is, and merely general recursive if the axioms
of~$\Th{T}$ are !!{decidable} but not primitive recursive.
