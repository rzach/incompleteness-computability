
\chapter{Derivations in Arithmetic Theories}

When we showed that all general recursive functions are representable
in~$\Th{Q}$, and in the proofs of the incompleteness theorems, we
claimed that various things are provable in $\Th{Q}$ and~$\Th{PA}$. The
proofs of this claims, however, just gave the arguments informally
without exhibiting actual derivations in natural deduction. We provide
some of these derivations in this capter.

For instance, in \olref[inc][req][bre]{lem:q-proves-add} we proved
that, for all $n$ and $m \in \Nat$, $\Th{Q} \Proves (\num{n} +
\num{m}) = \num{n+m}$. We did this by induction on $m$.

\begin{proof}[Proof of {\olref[inc][req][bre]{lem:q-proves-add}}]
Base case: $m = 0$. Then what has to be proved is that, for all $n$,
$\Th{Q} \Proves \num{n} + \num{0} = \num{n+0}$. Since $\num{0}$ is
just $\Obj 0$ and $\num{n+0}$ is $\num{n}$, this amounts to showing
that $\Th{Q} \Proves (\num{n} + \Obj 0) = \num{n}$. The derivation
\begin{prooftree}
  \AxiomC{$\lforall[x][(x + \Obj 0) = x]$}
  \RightLabel{\Elim\lforall}
  \UnaryInfC{$(\num{n} + \Obj 0) = \num{n}$}
\end{prooftree}
is a natural deduction derivation with one undischarged assumption,
and that undischarged assumption is an axiom of~$\Th{Q}$.

Inductive step: Suppose that, for any $n$, $\Th{Q} \Proves (\num{n} +
\num{m}) = \num{n+m}$ (say, by a derivation $\delta_n$). We have to show
that also $\Th{Q} \Proves (\num{n} + \num{m+1}) = \num{n+m+1}$. Note
that $\num{m+1}$ is the same as $\num{m}'$, and that $\num{n+m+1}$ is
the same as $\num{n+m}'$. So we are looking for a derivation of
$(\num{n} + \num{m}') = \num{n+m}'$ from the axioms of $\Th{Q}$. Our
derivation may use the derivation $\delta_n$ which exists by inductive
hypothesis.
\begin{prooftree}
  \AxiomC{}
  \RightLabel{$\delta_n$}
  \DeduceC{$(\num{n} + \num{m}) = \num{n+m}$}
  \AxiomC{$\lforall[x][\lforall[y][(x+y') = (x+y)']]$}
  \RightLabel{\Elim\lforall}
  \UnaryInfC{$\lforall[y][(\num{n}+y') = (\num{n}+y)']$}
  \RightLabel{\Elim\lforall}
  \UnaryInfC{$(\num{n}+\num{m}') = (\num{n}+\num{m})'$}
    \RightLabel{\Elim=}
    \BinaryInfC{$(\num{n}+\num{m}') = \num{n+m}'$}
\end{prooftree}
\end{proof}

In \olref[inc][req][min]{lem:less}, we showed that $\Th{Q} \Proves
\lforall[x][\lnot x < \Obj 0]$. What does an actual derivation look like?

\begin{proof}[Proof of {\olref[inc][req][min]{lem:less}}]
To prove a universal claim like this, we use $\Intro\lforall$, which
requires a derivation of $\lnot a < \Obj 0$. Looking at axiom $Q_8$,
this means proving $\lnot \exists z (z' + a) = \Obj 0$. Specifically,
if we had a proof of the latter, $Q_8$ would allow us to prove the
former (recall that $A \liff B$ is short for $(A \lif B) \land (B \lif
A)$.
\begin{prooftree}\footnotesize
  \AxiomC{$\lnot\lexists[z][(z' + a) = \Obj 0]$}
  \AxiomC{$\lforall[x][\lforall[y][(x < y \liff \lexists[z][(z' + x) = y])]]$}
  \RightLabel{\Elim\lforall}
  \UnaryInfC{$\lforall[y][(a < y \liff \lexists[z][(z' + a) = y])]$}
  \RightLabel{\Elim\lforall}
  \UnaryInfC{$a < \Obj 0 \liff \lexists[z][(z' + a) = \Obj 0]$}
  \RightLabel{\Elim\land}
  \UnaryInfC{$a < \Obj 0 \lif \lexists[z][(z' + a) = \Obj 0]$}
  \AxiomC{$\Discharge{a < \Obj 0}{1}$}
  \RightLabel{\Elim\lif}
  \BinaryInfC{$\lexists[z][(z' + a) = \Obj 0]$}
  \RightLabel{\Elim\lnot}
  \insertBetweenHyps{\hskip -3em}
  \BinaryInfC{$\lfalse$}
  \DischargeRule{\Intro\lnot}{1}
  \UnaryInfC{$\lnot a<\Obj 0$}
\end{prooftree}
This is a derivation of $\lnot a<\Obj 0$ from
$\lnot\lexists[z][(z' + a) = \Obj 0]$ (and~$Q_8$); let's call it~$\delta_1$

Ok, so how to prove $\lnot\lexists[z][(z' + a) = \Obj 0]$ from the
axioms of~$\Th{Q}$? To prove a negated claim like this, we'd need a
derivation of the form
\begin{prooftree}
  \AxiomC{$\Discharge{\lexists[z][(z' + a) = \Obj 0]}{2}$}
  \DeduceC{$\lfalse$}
  \DischargeRule{\Intro\lnot}{2}
  \UnaryInfC{$\lnot\lexists[z][(z' + a) = \Obj 0]$}
  \end{prooftree}
How do we get a contradiction from an existential claim? We introduce
a constant for the existentially quantified variable and use
\Elim\lexists:
\begin{prooftree}
  \AxiomC{$\Discharge{\lexists[z][(z' + a) = \Obj 0]}{2}$}
  \AxiomC{$\Discharge{(b'+a) = \Obj 0}{3}$}
  \RightLabel{$\delta_2$}
    \DeduceC{$\lfalse$}
    \DischargeRule{\Elim\exists}{3}
    \BinaryInfC{$\lfalse$}
  \DischargeRule{\Intro\lnot}{2}
  \UnaryInfC{$\lnot\lexists[z][(z' + a) = \Obj 0]$}
  \end{prooftree}
Now the task is to fill in $\delta_2$, i.e., prove $\lfalse$ from
$(b'+a) = \Obj 0$ and the axioms of~$\Th{Q}$. $Q_2$ says that $\Obj 0$
can't be the successor of some number, so one way of doing that would
be to show that $(b' + a)$ is equal to the successor of some
number. Since that expression itself is a sum, the axioms for addition
must come into play. If $\eq[a][\Obj 0]$, $Q_5$ would tell us that
$\eq[(b' + a)][b']$, i.e., $b' + a$ is the successor of some number,
namely of~$b$. On the other hand, if $\eq[a][c']$ for some $c$, then
$\eq[(b'+a)][(b'+c')]$ by \Elim\eq, and $\eq[(b'+c')][(b'+c)']$
by~$Q_6$. So again, $b'+a$ is the successor of a number---in this
case, $b'+c$. So the strategy is to divide the task into these two
cases. We also have to verify that $\Th{Q}$ proves that one of these
cases holds, i.e., $\Th{Q} \Proves a = 0 \lor \lexists[y][(a =
  y')]$. This can be proved from axiom~$Q_8$:
\begin{prooftree}\footnotesize
  \AxiomC{$\Discharge{\lnot (a = 0 \lor \lexists[y][a = y'])}{4}$}
  \AxiomC{$\lforall[x][(\lnot x=0 \lif  \lexists[y][x = y'])]$}
    \RightLabel{\Elim\forall}
    \UnaryInfC{$\lnot a = 0 \lif \lexists[y][a = y']$}
  \AxiomC{$\Discharge{\lnot (a = 0 \lor \lexists[y][a = y'])}{4}$}
  \AxiomC{$\Discharge{a=0}{5}$}
  \RightLabel{\Intro\lor}
  \UnaryInfC{$a = 0 \lor \lexists[y][a = y']$}
    \insertBetweenHyps{\hskip -.1em}
  \RightLabel{\Elim\lnot}
    \BinaryInfC{$\lfalse$}
    \DischargeRule{\Intro\lnot}{5}
    \UnaryInfC{$\lnot a = 0$}
        \insertBetweenHyps{\hskip -.5em}
    \RightLabel{\Elim\lif}
    \BinaryInfC{$\lexists[y][a = y']$}
  \RightLabel{\Intro\lor}
  \UnaryInfC{$a = 0 \lor \lexists[y][a = y']$}
  \insertBetweenHyps{\hskip -3.5em}
  \RightLabel{\Elim\lnot}
  \BinaryInfC{$\lfalse$}
  \DischargeRule{\FalseCl}{4}
  \UnaryInfC{$a = 0 \lor \lexists[y][a = y']$}
\end{prooftree}
Call this derivation~$\delta_3$. Here are the two cases:

Case 1: Prove $\lfalse$ from $\eq[a][\Obj 0]$ and $\eq[(b'+a)][\Obj
  0]$ (and axioms $Q_2$, $Q_5$):
\begin{prooftree}\footnotesize
  \AxiomC{$\lforall[x][\lnot \Obj 0 = x']$}
  \RightLabel{\Elim\lforall}
  \UnaryInfC{$\lnot \Obj 0 = b'$}
  \AxiomC{$\lforall[x][(x+\Obj 0) = x]$}
  \RightLabel{\Elim\lforall}
  \UnaryInfC{$(b' + \Obj 0) = b'$}
  \AxiomC{$a = \Obj 0$}
  \AxiomC{$(b'+a) = \Obj 0$}
  \RightLabel{\Elim=}
  \BinaryInfC{$(b' + \Obj 0) = \Obj 0$}
  \UnaryInfC{$\Obj 0 = (b' + \Obj 0)$}
      \insertBetweenHyps{\hskip -.5em}
  \RightLabel{\Elim=}
  \BinaryInfC{$\Obj 0 = b'$}
  \RightLabel{\Elim\lnot}
  \BinaryInfC{$\lfalse$}
\end{prooftree}
Call this derivation~$\delta_4$. (We've abbreviated the derivation of
$\Obj 0 = (b' + \Obj 0)$ from $(b' + \Obj 0) = \Obj 0$ by a double
inference line.)

Case 2: Prove $\lfalse$ from $\lexists[y][a = y']$ and
$\eq[(b'+a)][\Obj 0]$ (and axioms $Q_2$, $Q_6$). We first show how to
derive $\lfalse$ from $\eq[a][c']$ and $\eq[(b'+a)][\Obj 0]$.
\begin{prooftree}\footnotesize
  \AxiomC{$\lforall[x][\lnot \Obj 0 = x']$}
  \RightLabel{\Elim\lforall}
  \UnaryInfC{$\lnot \Obj 0 = (b'+c)'$}
  \AxiomC{$a = c'$}
  \AxiomC{$(b'+a) = \Obj 0$}
  \RightLabel{\Elim=}
  \BinaryInfC{$\Obj (b'+c') = \Obj 0$}
  \AxiomC{$\lforall[x][\lforall[y][(x+y') = (x+y)']]$}
  \RightLabel{\Elim\lforall}
  \UnaryInfC{$\lforall[y][(b'+y') = (b'+y)']$}
  \RightLabel{\Elim\lforall}
  \UnaryInfC{$(b' + c') = (b'+c)'$}
  \RightLabel{\Elim=}
  \BinaryInfC{$\Obj 0 = (b' + c)'$}
  \RightLabel{\Elim\lnot}
  \BinaryInfC{$\lfalse$}
\end{prooftree}
Call this $\delta_5$. We get the required derivation $\delta_6$ by
applying $\Elim\lexists$ and discharging the assumption $\eq[a][c']$:
\begin{prooftree}
  \AxiomC{$\lexists[y][a=y']$}
  \AxiomC{$\Discharge{a = c'}{6} \quad \eq[(b'+a)][\Obj 0]$}
  \RightLabel{$\delta_5$}
  \DeduceC{$\lfalse$}
  \DischargeRule{\Elim\exists}{6}
  \BinaryInfC{$\lfalse$}
\end{prooftree}

  
Putting everything together, the full proof looks like this:
\begin{prooftree}\footnotesize
  \AxiomC{$\Discharge{\lexists[z][(z' + a) = \Obj 0]}{2}$}
  \AxiomC{}
  \RightLabel{$\delta_3$}
  \DeduceC{$a = 0 \lor \lexists[y][(a = y')]$}
  \AxiomC{$\begin{gathered}
      \Discharge{a = \Obj 0}{7} \\
      \Discharge{(b'+a) = \Obj 0}{3}
      \end{gathered}$}
    \RightLabel{$\delta_4$}
    \DeduceC{$\lfalse$}
    \AxiomC{$\begin{gathered}
        \Discharge{\lexists[y][a=y']}{7} \\
        \Discharge{(b'+a) = \Obj 0}{3}
        \end{gathered}$\quad}
    \RightLabel{$\delta_6$}
    \DeduceC{$\lfalse$}
    \DischargeRule{\Elim\lor}{7}
    \insertBetweenHyps{\hskip -.5em}
    \TrinaryInfC{$\lfalse$}
    \RightSubproofLabel{$\delta_2$}
    \DischargeRule{\Elim\exists}{3}
    \BinaryInfC{$\lfalse$}
  \DischargeRule{\Intro\lnot}{2}
  \UnaryInfC{$\lnot\lexists[z][(z' + a) = \Obj 0]$}
  \RightLabel{$\delta_1$}
  \DeduceC{$\lnot a<\Obj 0$}
  \RightLabel{\Intro\lforall}
  \UnaryInfC{$\lforall[x][\lnot x < \Obj 0]$}
\end{prooftree}\qedhere
\end{proof}

In the proof of \olref[inc][inp][ros]{thm:rosser}, we defined
$\ORProv(y)$ as \[\lexists[x][(\OPrf(x, y) \land \lforall[z][(z < x
    \lif \lnot \ORefut(z, y))])].\] $\OPrf(x,y)$ is the formula
representing the proof relation of~$\Th{T}$ (a consistent,
axiomatizable extension of~$\Th{Q}$) in $\Th{Q}$, and $\ORefut(z, y)$
is the formula representing the refutation relation. That means that
if $n$ is the G\"odel number of a proof of~$!A$, then $\Th{Q} \Proves
\OPrf(\num{n}, \gn{!A})$, and otherwise $\Th{Q} \Proves
\lnot\OPrf(\num{n}, \gn{!A})$. Similarly, if $n$ is the G\"odel number
of a proof of $\lnot !A$, then $\Th{Q} \Proves \ORefut(\num{n},
\gn{!A})$, and otherwise $\Th{Q} \Proves \lnot\ORefut(\num{n},
\gn{!A})$. We use the Diagonal Lemma to find a sentence $!R$ such that
$\Th{Q} \Proves !R \liff \lnot \ORProv(\gn{!R})$. Rosser's Theorem
states that $\Th{T} \Proves/ !R$ and $\Th{T} \Proves/ \lnot !R$. Both
claims were proved indirectly: we show that if $\Th{T} \Proves !R$,
$\Th{T}$ is inconsistent, i.e., $\Th{T} \Proves \lfalse$, and the same
if $\Th{T} \Proves \lnot !R$. 

\begin{proof}[Proof of {\olref[inc][inp][ros]{thm:rosser}}]
First we prove something things about~$<$. By
\olref[inc][req][min]{lem:less}, we know that $\Th{Q} \Proves
\lforall[x][(x < \num {n+1} \lif (\eq[x][\Obj 0] \lor \dots \lor
  \eq[x][\num n]))]$ for every~$n$. So of course also (if $n>1$),
$\Th{Q} \Proves \lforall[x][(x < \num {n} \lif (\eq[x][\Obj 0] \lor
  \dots \lor \eq[x][\num{n-1}]))]$. We can use this to derive
$\eq[a][\Obj 0] \lor \dots \lor \eq[a][\num{n-1}]$ from $a < \num{n}$:
\begin{prooftree}
  \AxiomC{$a < \num{n}$}
  \AxiomC{}
  \DeduceC{$\lforall[x][(x < \num{n} \lif (x = \num{0} \lor 
      \dots \lor x = \num{n-1}))]$}
  \RightLabel{\Elim\forall}
  \UnaryInfC{$a < \num{n} \lif (a = \num{0} \lor 
      \dots \lor a = \num{n-1})$}
  \RightLabel{\Elim\lif}
  \BinaryInfC{$a = \num{0} \lor \dots \lor a = \num{n-1}$}
\end{prooftree}
Let's call this derivation $\lambda_1$.

Now, to show that $\Th{T} \Proves/ !R$, we assume that $\Th{T}
\Proves !R$ (with a derivation~$\delta$) and show that $\Th{T}$ then
would be inconsistent. Let $n$ be the G\"odel number
of~$\delta$. Since $\OPrf$ represents the proof relation in~$\Th{Q}$,
there is a derivation~$\delta_1$ of $\OPrf(\num{n},
\gn{!R})$. Furthermore, no $k < n$ is the G\"odel number of a
refutation of~$!R$ since $\Th{T}$ is assumed to be consistent, so for
each $k < n$, $\Th{Q} \Proves \lnot \ORefut(\num{k}, \gn{!R})$; let
$\rho_k$ be the corresponding derivation. We get a derivation of
$\ORProv(\gn{!R})$:
\begin{prooftree}\footnotesize
  \AxiomC{}
  \RightLabel{$\delta_1$}
  \DeduceC{$\OPrf(\num{n}, \gn{!R})$}

  \AxiomC{$\Discharge{a < \num{n}}{1}$}
  \RightLabel{$\lambda_1$}
  \DeduceC{$\begin{gathered}[b]
      a= \num{0} \lor \dots {} \\
      {} \lor a = \num{n-1}
      \end{gathered}$}
  \AxiomC{$\dots$}

  \AxiomC{$\Discharge{a=\num{k}}{2}$}
  \AxiomC{}
  \RightLabel{$\rho_k$}
  \DeduceC{$\lnot \ORefut(\num{k}, \gn{!R})$}
  \RightLabel{\Elim=}
  \BinaryInfC{$\lnot \ORefut(a, \gn{!R})$}
  \AxiomC{$\dots$}
  \DischargeRule{$\Elim\lor^*$}{2}
  \doubleLine
  \insertBetweenHyps{\hskip -1pt}
  \QuaternaryInfC{$\lnot \ORefut(a, \gn{!R})$}
  \DischargeRule{\Intro\lif}{1}
  \UnaryInfC{$a < \num{n} \lif \lnot \ORefut(a, \gn{!R})$}
  \RightLabel{\Intro\lforall}
  \UnaryInfC{$\lforall[z][(z < \num{n} \lif \lnot \ORefut(z, \gn{!R}))]$}

    \insertBetweenHyps{\hskip -5pt}
  \RightLabel{\Intro\land}
  \BinaryInfC{$\OPrf(\num{n}, \gn{!R}) \land \lforall[z][(z < \num{n} \lif \lnot \ORefut(\num{z}, \gn{!R}))]$}
  \RightLabel{\Intro\lexists}
  \UnaryInfC{$\lexists[x][(\OPrf(x, \gn{!R}) \land \lforall[z][(z < x \lif \lnot \ORefut(z, \gn{!R}))])]$}
\end{prooftree}
(We abbreviate multiple applications of $\Elim\lor$ by $\Elim\lor^*$
above.)  We've shown that if $\Th{T} \Proves !R$ there would be a
derivation of~$\ORProv(\Gn{!R})$.  Then, since $\Th{T} \Proves R \liff
\lnot \ORProv(\gn{!R})$, also $\Th{T} \Proves \ORProv(\gn{!R}) \lif
\lnot R$, we'd have $\Th{T} \Proves \lnot !R$ and $\Th{T}$ would be
inconsistent.

Now let's show that $\Th{T} \Proves/ \lnot !R$. Again, suppose it
did. Then there is a derivation $\rho$ of $\lnot !R$ with G\"odel
number $m$---a refutation of~$!R$---and so $\Th{Q} \Proves
\ORefut(\num{m}, \gn{!R})$ by a derivation~$\rho_1$. Since we assume
$\Th{T}$ is consistent, $\Th{T} \Proves/ !R$, and so for all $k$, $k$ is
not a G\"odel number of a derivation of~$!R$, and hence $\Th{Q} \Proves \lnot
\OPrf(\num{k}, \gn{!R})$ by a derivation $\pi_k$. So we have:
  
\begin{prooftree}\footnotesize
  \AxiomC{}
  \RightLabel{$\lambda_2$}
  \DeduceC{$\begin{gathered}[b] a = \num{0} \lor \dots \lor\\
      a = \num{m} \lor \num{m} < a\end{gathered}$}
  \AxiomC{\dots}
  \AxiomC{$\begin{gathered}\Discharge{\OPrf(a, \gn{!R})}{1}\\
    \Discharge{a = \num{k}}{2}\end{gathered}$}
  \RightLabel{$\pi_k'$}
  \DeduceC{$\lfalse$}
  \RightLabel{$\lfalse_I$}
  \UnaryInfC{$\num{m}<a$}
  \AxiomC{\dots}
  \AxiomC{$\Discharge{\num{m} < a}{2}$}
  \DischargeRule{$\Elim\lor^*$}{2}
  \insertBetweenHyps{\hskip 5pt}
  \doubleLine
  \QuinaryInfC{$\num{m} < a$}
  \AxiomC{}
  \RightLabel{$\rho_1$}
  \DeduceC{$\ORefut(\num{m}, \gn{!R})$}
  \RightLabel{\Intro\land}
  \BinaryInfC{$\num{m}
    < a \land \ORefut(\num{m}, \gn{!R})$}
  \RightLabel{\Intro\exists}
  \UnaryInfC{$\exists z(z
    < a \land \ORefut(z, \gn{!R}))$}
  \DischargeRule{\Intro\lif}{1}
  \UnaryInfC{$\OPrf(a, \gn{!R}) \lif \exists z(z
    < a \land \ORefut(z, \gn{!R}))$}
  \RightLabel{\Intro\forall}
  \UnaryInfC{$\forall x(\OPrf(x, \gn{!R}) \lif \exists z(z
    < x \land \ORefut(z, \gn{!R})))$}
  \DeduceC{$\lnot\exists x(\OPrf(x, \gn{!R}) \land \forall z(z
    < x \lif \lnot \ORefut(z, \gn{!R})))$}
\end{prooftree}
where $\pi_k'$ is the derivation
\begin{prooftree}
  \AxiomC{}
  \RightLabel{$\pi_k$}
  \DeduceC{$\lnot \OPrf(\num{k}, \gn{!R})$}
  \AxiomC{$a = \num{k}$}
  \AxiomC{$\OPrf(a, \gn{!R})$}
  \RightLabel{\Elim=}
  \BinaryInfC{$\OPrf(\num{k}, \gn{!R})$}
  \RightLabel{\Elim\lnot}
  \BinaryInfC{$\lfalse$}
\end{prooftree}
and $\lambda_2$ is
\begin{prooftree}\footnotesize
  \AxiomC{}
  \RightLabel{$\lambda_3$}
  \DeduceC{$\begin{gathered}[b]a < \num{m} \lor \\a = \num{m} \lor\\ \num{m} < a\end{gathered}$}

  \AxiomC{$\Discharge{a < \num{m}}{3}$}
  \RightLabel{$\lambda_1$}
  \DeduceC{$a = \num{0} \lor \dots \lor a = \num{m-1}$}
  \RightLabel{\Intro\lor}
  \UnaryInfC{$\begin{gathered}[b]
      a = \num{0} \lor \dots \lor \\
      a = \num{m} \lor \num{m} < a
    \end{gathered}$}

  \AxiomC{$\Discharge{a = \num{m}}{3}$}
  \RightLabel{\Intro\lor}
  \UnaryInfC{$\begin{gathered}[b]
      a = \num{0} \lor \dots \lor \\
      a = \num{m} \lor \num{m} < a
    \end{gathered}$}
  
  \AxiomC{$\Discharge{\num{m} < a}{3}$}
  \RightLabel{\Intro\lor}
  \UnaryInfC{$\begin{gathered}[b]
      a = \num{0} \lor \dots \lor \\
      a = \num{m} \lor \num{m} < a
    \end{gathered}$}

  \insertBetweenHyps{\hskip 5pt}
  \doubleLine
  \DischargeRule{$\Elim\lor^*$}{3}
  \QuaternaryInfC{$a = \num{0} \lor \dots \lor a = \num{m} \lor \num{m} < a$}
\end{prooftree}
(The derivation $\lambda_3$ exists by \olref[inc][req][min]{lem:trichotomy}.)
\end{proof}

\OLEndChapterHook


