% phil479-print.tex

% driver file phil479-print.tex to produce the Logic III textbook on
% with same type block as in printed version, but but with on-screen
% features (color, links, etc).

% We use the memoir class for maximal flexibility of layout, but any
% class will do

\documentclass[11pt,openany]{memoir}

\RequirePackage{amsthm}
\RequirePackage{xcolor}
\RequirePackage{mdframed}
\RequirePackage[full]{textcomp}
\RequirePackage{gitinfo2}

\RequirePackage[pdftex,breaklinks,bookmarks,bookmarksopen,bookmarksopenlevel=1,colorlinks,urlcolor=dark-gray,linkcolor=reflex-blue]{hyperref}

%\usepackage[nonumberlist,toc,style=index]{glossaries}
%\makeglossaries


% let's set the whole thing in Baskervald X, with Universalis ADF
% Standard for sans-serif, and spread the lines a bit to make the text
% more readable

\usepackage[sfdefault]{universalis}
\usepackage[osf]{Baskervaldx} % oldstyle figures
\usepackage[bigdelims,baskervaldx]{newtxmath} 
\usepackage[cal=boondoxo]{mathalfa}

% Make sure we have a copyright symbol

\def\copyright{\textcircled{C}}

\def\oljobname{phil479}

% next line needed for Adobe Acrobat to open the PDF.  may have
% somethign to do with transparency in the PNG graphics files?

\pdfminorversion=4

% set stock & paper size to Lulu's Crown Quarto dimensions

\setstocksize{24.589cm}{18.91cm}

\settrimmedsize{\stockheight}{\stockwidth}{*}
\settrims{0pt}{0pt}

% let's calculate the line length for 65 characters in \normalfont

\setlxvchars

% set the size of the type block to golden ratio calculated width

\settypeblocksize{*}{1.05\lxvchars}{1.62}

% set spine and and edge margin

\setlrmargins{*}{*}{1.3}
\setulmargins{90pt}{*}{*}
\setheaderspaces{*}{*}{1}

\checkandfixthelayout

% Chapter style

\makeatletter

\colorlet{leadbeater}{black}
\colorlet{dkleadbeater}{black}
\colorlet{ltleadbeater}{black!5}
\newlength{\barlength}
\makechapterstyle{leadbeater}{%
  \setlength{\afterchapskip}{40pt}
  \setlength{\beforechapskip}{50pt}
    \setlength{\midchapskip}{10pt}
  \renewcommand*{\afterchapternum}{\par\nobreak\vskip 0pt}
  \renewcommand*{\chapnamefont}{\fontsize{14pt}{0pt}\selectfont\bfseries\sffamily}
  \renewcommand*{\chapnumfont}{\fontsize{14pt}{0pt}\selectfont\bfseries\sffamily}
  \renewcommand*{\chaptitlefont}{\normalfont\fontsize{48pt}{48pt}\selectfont\bfseries\itshape\color{leadbeater}}
  \renewcommand*{\printchaptername}{%
    \chapnamefont\MakeUppercase{\@chapapp}}
  \renewcommand*{\printchaptertitle}[1]{%
    \chaptitlefont ##1\\[-\baselineskip]%
    \hspace*{-20pt}%
    \smash{\color{leadbeater}\rule{7pt}{300pt}}}
}

\renewcommand*{\partnamefont}{\fontsize{24pt}{0pt}\selectfont\bfseries\sffamily}
\renewcommand*{\partnumfont}{\fontsize{24pt}{0pt}\selectfont\bfseries\sffamily}
\renewcommand*{\parttitlefont}{\normalfont\fontsize{54pt}{54pt}\selectfont\bfseries\itshape\color{leadbeater}}
\renewcommand*{\printpartname}{%
  \partnamefont PART}

\makeatother

\chapterstyle{leadbeater}

\copypagestyle{leadbeater}{headings}

\makeoddhead{leadbeater}{\small\sffamily\color{leadbeater}\rightmark}{}
            {\color{leadbeater}\sffamily\bfseries\thepage}

\makeevenhead{leadbeater}{\small\sffamily\color{leadbeater}\leftmark}{}
            {\color{leadbeater}\sffamily\bfseries\thepage}

\usepackage{atbegshi}  % to kill the blank page after the titling page

\usepackage[font={small,it}]{caption}


% \olpath has to point to the location of the OLP main
% directory/folder.  We're compiling from subdirectory
% courses/phil379/, so the main directory is two
% levels up.

\newcommand{\olpath}{../../}

% load all the Open Logic definitions. This will also load the
% local definitions in open-logic-sample-config.sty

\input{\olpath/sty/open-logic.sty}

% we want all the problems deferred to the very end

\input{\olpath/sty/open-logic-defer.sty}

% use high-res versions of photos

\tagfalse{olphotos-lowres}

% all links plain black for printing

\hypersetup{hidelinks,
  pdftitle={Incompleteness and Computability},
  pdfauthor = {Open Logic Project}
}

% load glossary entries

%\loadglsentries{include/glossary}

% end preamble

\input{includeonly}

\begin{document}

% Now load the actual text

% phil379.tex
%
% driver file phil379.tex to produce text

\preto\OLEndChapterHook{\IfFileExists{include/summary-\thechapter}
        {\section*{Summary}\addcontentsline{toc}{section}{Summary}
        \let\emph\textbf\input{include/summary-\thechapter}\let\emph\textit}{}}

\problemsperchapter
\allowdisplaybreaks

\frontmatter

% bastard title

\pagestyle{empty}

\vspace*{100pt}

\begin{raggedleft}

{\fontsize{24pt}{24pt}\selectfont\bfseries\sffamily%
Incompleteness and
Computability}

\end{raggedleft}


\newpage

% editorial board

\vspace*{100pt}

{\bfseries\itshape The Open Logic Project}

\bigskip

\textbf{\color{leadbeater}Instigator}

\medskip

Richard Zach, \emph{University of Calgary}

\bigskip

\textbf{\color{leadbeater}Editorial Board}

\medskip

Aldo Antonelli,$^\dagger$ \emph{University of California, Davis}

Andrew Arana, \emph{Universit\'e Paris I Panth\'enon--Sorbonne}

Jeremy Avigad, \emph{Carnegie Mellon University}

Walter Dean, \emph{University of Warwick}

Gillian Russell, \emph{University of North Carolina}

Nicole Wyatt, \emph{University of Calgary}

Audrey Yap, \emph{University of Victoria}

\bigskip

\textbf{\color{leadbeater}Contributors}

\medskip

Samara Burns, \emph{University of Calgary}

Dana H\"agg, \emph{University of Calgary}

\newpage

% title

\vspace*{100pt}

\begin{raggedleft}

{\fontsize{24pt}{24pt}\selectfont\bfseries\sffamily%
  Incompleteness and Computability}

\smallskip

{\fontsize{18pt}{18pt}\selectfont\bfseries\itshape An Open Logic Text}

\vspace{100pt}

\fontsize{14pt}{14pt}\selectfont Remixed by Richard Zach

\vfill

\textsc{Spring 2017} \textit{bis}

\end{raggedleft}


\newpage

% credits

\noindent
The Open Logic Project would like to acknowledge the generous support
of the \href{http://arts.ucalgary.ca/}{Faculty of Arts} and
the \href{http://www.ucalgary.ca/taylorinstitute/}{Taylor Institute of
Teaching and Learning} of the University of Calgary,
and the \href{http://albertaoer.com}{Alberta Open Educational Resources
(ABOER) Initiative}, which is made possible through an investment from
the Alberta government.



\bigskip

\noindent\includegraphics[height=1cm]{assets/ucarts-color}

\medskip

\noindent\includegraphics[height=1cm]{assets/ti-color}

\medskip

\noindent\includegraphics[width=4cm]{assets/aboer-color}

\vfill

\noindent Cover illustrations by \href{http://mattleadbeater.com}{Matthew
  Leadbeater}, used under a
\href{http://creativecommons.org/licenses/by-nc/4.0/}{Creative Commons
  Attribution-NonCommercial 4.0 International License}.

\vfill

\noindent Typeset in Baskervald X and Universalis ADF Standard by
\LaTeX.

\vfill

\noindent This version of \texttt{phil479} is revision \gitAbbrevHash{}
(\gitAuthorDate), with content generated
from \texttt{OpenLogicProject} revision
\input{olprevision}

\vfill

% oluselicense generates a license mark that a) licenses the result
% under a CC-BY licence and b) acknowledges the original source (the
% OLP).  Acknowledgment of the source is a requirement under the
% conditions of the CC-BY license used by the OLP, but you are not
% required to license the product itself under CC-BY.

\renewcommand{\ollicensefont}{\fontsize{8.5pt}{11pt}\selectfont}

\noindent
\oluselicense
% Title of this version of the OLT with link to source
{\href{https://github.com/rzach/phil479}{\textit{Incompleteness and
Computability}}}
% Author of this version
{\href{http://richardzach.org/}{Richard Zach}}

\newpage
\pagestyle{leadbeater}
\currentpdfbookmark{Table of Contents}{name}
\tableofcontents*

%% !TeX root = ../ic-screen.tex

\chapter{Preface}

G\"odel's incompleteness theorems are some of the most celebrated
results in mathematical logic, if not in mathematics. The first of
these states, roughly, that axiomatized mathematical theories that can
carry out a minimal amount of arithmetic are either inconsistent
(trivial, prove everything) or incomplete. In other words, if a
mathematical theory can be written down in a compact way (is
axiomatized), is strong enough to state and prove some basic facts
about natural numbers, and contains no contradictions that would
render it useless, there are always statements in the language of the
theory it doesn't settle, i.e., sentences~$A$ such that the theory
proves neither $A$ nor~$\lnot A$. (This is the first incompleteness
theorem.)

This result was historically surprising since it might be
taken to mean that we can never write down an axiomatic theory that
``captures'' all mathematical truths---at least if we require that
what follows from the theory must be verified by a finite derivation
or proof which we can mechanically test for correctness. One
consequence of the result is that mathematical truth is undecidable:
there cannot be a mechanical way to decide, given a statement in a
mathematical theory, whether it follows from the axioms. Another
consequence is that mathematical theories that are strong enough and
consistent cannot \emph{prove} their own consistency---at least for
the most straightforward way of formalizing the statement of the
theory's consistency in the theory itself. (This is the second
incompleteness theorem.)

Assuming a minimal background of formal logic on the part of the
reader, it is actually not hard to state and prove the first two of
these results. In fact, we do so below in \cref{inc:int::chap}. But
the version of the result we prove makes stronger assumptions for the
first incompleteness than actually needed. We also do not show that
these assumptions hold for the theories discussed, and the proofs are
not constructive: they show that theories are incomplete, but don't
give examples of sentences that the theory leaves undecided. To do
this in detail requires more background.

In \cref{cmp:rec::chap,inc:art::chap,inc:req::chap} we provide this
background: we discuss a model of computability (the primitive
recursive functions), we show that by assigning numbers to symbols we
can ``arithmetize'' the syntax and proof theory of arithmetical
theories using primitive recursive functions and relations, and
finally that the very simple arithmetical theory~$\Th{Q}$
``represents'' all primitive recursive functions and relations.

With this background it is then possible to state and prove the
incompleteness theorems in the same level of detail as any thorough
mathematical exposition of these results would. In
\cref{inc:inp::chap} we prove G\"odel's original version of the first
incompleteness theorem, Rosser's improved version, the second
incompleteness theorem, as well as L\"ob's theorem and Tarski's
theorem about the undefinability of truth.

The incompleteness phenomena are closely tied to the notion of
computability. The incompleteness theorem applies to theories that can
be computably generated from computable sets of axioms. The mechanics
of the arithmetization of syntax requires a model of computation
(recursive functions) to spell out the details. And the result itself
is structurally related to a famous result from the theory of
computability, namely the theorem that the halting problem is
undecidable due to Church and Turing. So it is natural to give an
alternative description and proof of the incompleteness theorem that
makes use of computability theory. We do this in \cref{comp-inc:chap}
after introducing the basic theory of partial computable functions
and decidable and computably enumerable sets.

That theories of arithmetic are incomplete means that they have models
that not only don't look like the ``standard model'' of the natural
numbers, but that make sentences false which are true in the standard
model. The structure of such models is itself an interesting area of
research. We provide a brief glimpse of it in \cref{mod:chap}.

The incompleteness theorems concern, in the first instance, theories
formulated and axiomatized in first-order languages, and for which we
assume the usual first-order consequence and provability relation. By
G\"odel's completeness theorem, we know that the proof systems we have
for first-order logic and theories axiomatized in it are as strong as
we want them to be: they prove everything that follows from the
axioms. In a sense, the incompleteness theorems say that it is exactly
this feature of first-order logic that prevents us from writing down
mathematical theories that ``settle every question.'' A natural way
out would be to adopt a stronger logic with a stronger consequence
relation which does: second-order logic. We discuss its most important
properties in \cref{sol:chap}: it is strong enough to characterize
arithmetical truth and it is much more expressive than first-order
logic. Second-order arithmetic is complete. But many results that hold
for first-order logic (such as the compactness and L\"owenheim--Skolem
theorems) fail for second-order logic.

This book introduces recursive functions explicitly as a model of
computability. Representability in $\Th{Q}$ can also be taken as a
model of computability, and along the way we (almost) show that it is
equivalent to recursive functions. The companion book
\href{https://slc.openlogicproject.org/}{\emph{Sets, Logic, Computation}}
discusses the Turing machine model of computation. In
\cref{lambda:chap}, we introduce yet another model of computability:
the (untyped) lambda-calculus.

\section*{Notes for Instructors}

This is a textbook on G\"odel's incompleteness theorems and recursive
function theory. I use it as the main text when I teach Philosophy 479
(Logic III) at the University of Calgary. It is based on material from the
\href{https://openlogicproject.org}{Open Logic Project}.

As its name suggests, the course is the third in a sequence, so
students (and hence readers of this book) are expected to be familiar
with first-order logic already. (Logic~I uses the text
\href{https://forallx.openlogicproject.org}{\emph{forall x: Calgary}},
and Logic~II another textbook based on the OLP,
\href{https://slc.openlogicproject.org}{\emph{Sets, Logic,
Computation}}.) The material assumed from Logic~II, however, is
included as \cref{fol:chap,nd:chap}, and it is not absolutely
necessary to assume more than that as background for a course based on
this book.

Logic III is a thirteen-week course, meeting three hours per week.
This is typically enough to cover the material in
\cref{inc:int::chap,cmp:rec::chap,inc:art::chap,inc:req::chap,inc:inp::chap}
and one or two of \cref{comp-inc:chap,mod:chap,sol:chap,lambda:chap},
depending on student interest. You may want to spend more time on the
basics of first-order logic and especially on natural deduction, if
students are not already familiar with it. Note that when provability
in arithmetical theories (such as $\Th{Q}$ and $\Th{PA}$) is discussed
in the main text, the proofs of provability claims are not given using
a specific !!{derivation} system. Rather, that certain claims follow
from the axioms by first-order logic is justified intuitively.
However, \cref{deriv:chap} contains a number of examples of actual
natural deduction !!{derivation}s from the axioms of~$\Th{Q}$.
\Cref{inc:art::chap} carries out the arithmetization of syntax for
natural deduction. This is a perhaps unique feature of this book; most
other tests just do it for axiomatic derivations. Those are much
easier to code, but much harder to give proofs with.

\section*{Acknowledgments}

The material in the OLP used in
\cref{inc:int::chap,cmp:rec::chap,inc:art::chap,inc:req::chap,inc:inp::chap,comp-inc:chap,lambda:chap}
was based originally on Jeremy Avigad's lecture notes on
``Computability and Incompleteness,'' which he contributed to the OLP.
I have heavily revised and expanded this material. The lecture notes,
e.g., based theories of arithmetic on an axiomatic !!{derivation} system. Here,
we use Gentzen's standard natural deduction system (described in
\cref{nd:chap}), which requires dealing with trees primitive
recursively (in \cref{cmp:rec:tre:sec}) and a more complicated
approach to the arithmetization of !!{derivation}s (in
\cref{inc:art:pnd:sec}). The material in \cref{lambda:chap} was also
expanded by Zesen Qian during his stay in Calgary as a Mitacs summer
intern.

The material in the OLP on model theory and models of arithmetic in
\cref{mod:chap} was originally taken from Aldo Antonelli's lecture
notes on ``The Completeness of Classical Propositional and Predicate
Logic,'' which he contributed to the OLP before his untimely death in
2015.

The biographies of logicians in \cref{bios:chap} and much of the
material in \cref{nd:chap} are originally due to Sam Burns.
Dana H\"agg originally worked on the material in \cref{fol:chap}.


\mainmatter
\pagestyle{leadbeater}

\olimport*[incompleteness/introduction]{introduction}

\olimport*[computability/recursive-functions]{recursive-functions}

\olimport*[incompleteness/arithmetization-syntax]{arithmetization-syntax}

\olimport*[incompleteness/representability-in-q]{representability-in-q}

\olimport*[incompleteness/incompleteness-provability]{incompleteness-provability}

\chapter{Models of Arithmetic}

\olimport*[model-theory/models-of-arithmetic]{introduction}

\olimport*[model-theory/basics]{reducts-and-expansions}

\olimport*[model-theory/basics]{isomorphism}

\olimport*[model-theory/basics]{theory-of-m}

\olimport*[model-theory/models-of-arithmetic]{standard-models}

\olimport*[model-theory/models-of-arithmetic]{non-standard-models}

\olimport*[model-theory/models-of-arithmetic]{models-of-q}

\olimport*[model-theory/models-of-arithmetic]{models-of-pa}

\olimport*[model-theory/models-of-arithmetic]{computable-models}

\OLEndChapterHook

\chapter{Second-Order Logic}

\olimport*[second-order-logic/syntax-and-semantics]{introduction}

\let\oldolsection\olsection
\let\olsection\nosection
\olimport*[second-order-logic/sol-and-set-theory]{introduction}
\let\olsection\oldolsection

\olimport*[second-order-logic/syntax-and-semantics]{terms-formulas}

\olimport*[second-order-logic/syntax-and-semantics]{satisfaction}

\olimport*[second-order-logic/syntax-and-semantics]{semantic-notions}

\olimport*[second-order-logic/syntax-and-semantics]{expressive-power}

\olimport*[second-order-logic/syntax-and-semantics]{inf-count}

\olimport*[second-order-logic/metatheory]{second-order-arithmetic}

\olimport*[second-order-logic/metatheory]{undecidability-and-axiomatizability}

\olimport*[second-order-logic/metatheory]{compactness}

\olimport*[second-order-logic/metatheory]{loewenheim-skolem}

\olimport*[second-order-logic/sol-and-set-theory]{comparing-sets}

\olimport*[second-order-logic/sol-and-set-theory]{cardinalities}

\olimport*[second-order-logic/sol-and-set-theory]{power-of-continuum}

\OLEndChapterHook

\appendix

\let\intro\comment
\let\endintro\endcomment

\chapter{First-order Logic}

\olimport*[first-order-logic/syntax-and-semantics]{first-order-languages}

\olimport*[first-order-logic/syntax-and-semantics]{terms-formulas}

\olimport*[first-order-logic/syntax-and-semantics]{free-vars-sentences}

\olimport*[first-order-logic/syntax-and-semantics]{substitution}

\olimport*[first-order-logic/syntax-and-semantics]{structures}

\olimport*[first-order-logic/syntax-and-semantics]{satisfaction}

\olimport*[first-order-logic/syntax-and-semantics]{semantic-notions}

\section{Theories}

\begin{defn}
A set of !!{sentence}s~$\Gamma$ is \emph{closed} iff, whenever
$\Gamma \Entails !A$ then $!A \in \Gamma$.  The \emph{closure} of a set
of !!{sentence}s~$\Gamma$ is $\Setabs{!A}{\Gamma \Entails !A}$.

We say that~$\Gamma$ is \emph{axiomatized by} a set of
sentences~$\Delta$ if $\Gamma$ is the closure of~$\Delta$
\end{defn}

\begin{ex}
The theory of strict linear orders in the language~$\Lang L_<$ is
axiomatized by the set
\begin{align*}
& \lforall[x][\lnot x < x], \\
& \lforall[x][\lforall[y][((x < y \lor y <
    x) \lor x = y)]], \\
& \lforall[x][\lforall[y][\lforall[z][((x < y
      \land y < z) \lif x < z)]]]
\end{align*}
It completely captures the intended !!{structure}s: every strict
linear order is a model of this axiom system, and vice versa, if $R$
is a linear order on a set $X$, then the structure $\Struct M$ with
$\Domain M = X$ and $\Assign{<}{M} = R$ is a model of this theory.
\end{ex}

\begin{ex}
The theory of groups in the language $\Obj 1$ (!!{constant}), $\cdot$
(two-place !!{function}) is axiomatized by
\begin{align*}
& \lforall[x][(x \cdot \Obj 1) = x]\\
& \lforall[x][\lforall[y][\lforall[z][\eq[(x \cdot (y \cdot z))][((x
          \cdot y) \cdot z)]]]]\\
& \lforall[x][\lexists[y][(x \cdot y) = \Obj 1]]
\end{align*}
\end{ex}

\begin{ex}
The theory of Peano arithmetic is axiomatized by the following
sentences in the language of arithmetic~$\Lang L_A$.
\begin{align*}
& \lnot\lexists[x][x' = \Obj 0]\\
& \lforall[x][\lforall[y][(x' = y' \lif x = y)]]\\
& \lforall[x][\lforall[y][(x < y \liff \lexists[z][(x + z' = y)])]]\\
& \lforall[x][\eq[(x + \Obj 0)][x]]\\
& \lforall[x][\lforall[y][\eq[(x + y')][(x + y)']]]\\
& \lforall[x][\eq[(x \times \Obj 0)][\Obj 0]]\\
& \lforall[x][\lforall[y][\eq[(x \times y')][((x \times y) + x)]]]\\
\intertext{plus all sentences of the form}
& (!A(\Obj 0) \land \lforall[x][(!A(x) \lif !A(x'))]) \lif \lforall[x][!A(x)]
\end{align*}
Since there are infinitely many sentences of the latter form, this
axiom system is infinite.  The latter form is called the
\emph{induction schema}. (Actually, the induction schema is a bit more
complicated than we let on here.)

The third axiom is an \emph{explicit definition} of~$<$.
\end{ex}

\OLEndChapterHook

\chapter{Natural Deduction}

\olimport*[first-order-logic/proof-systems]{natural-deduction}[\nosection]

\olimport*[first-order-logic/natural-deduction]{rules-and-proofs}

\olimport*[first-order-logic/natural-deduction]{proving-things}

\olimport*[first-order-logic/natural-deduction]{identity}

\olimport*[first-order-logic/natural-deduction]{proof-theoretic-notions}

\OLEndChapterHook


\stopproblems
\def\ifproblems#1{}

\def\captionnamefont{\sffamily\color{leadbeater}}
\def\captiontitlefont{\sffamily\color{leadbeater}}

\def\figurename{Fig.}
\setlength{\olphotowidth}{.45\textwidth}

\chapter{Biographies}

\olimport*[history/biographies]{alonzo-church}

\olimport*[history/biographies]{kurt-goedel}

\olimport*[history/biographies]{rozsa-peter}

\olimport*[history/biographies]{julia-robinson}

\olimport*[history/biographies]{alfred-tarski}

\backmatter

\clearpage
\photocredits

\bibliographystyle{\olpath/bib/natbib-oup}

\bibliography{\olpath/bib/open-logic.bib}

\olimport*{\olpath/content/open-logic-about}



\end{document}

